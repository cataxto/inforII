\documentclass{article}
\usepackage[utf8]{inputenc}
\usepackage[spanish]{babel}
\usepackage{listings}
\usepackage{graphicx}
\graphicspath{ {images/} }
\usepackage{cite}

\begin{document}

\begin{titlepage}
    \begin{center}
        \vspace*{1cm}
            
        \Huge
        \textbf{Taller de Memoria}
            
        \vspace{0.5cm}
        \LARGE
        Subtítulo
            
        \vspace{1.5cm}
            
        \textbf{María Catalina Hernández Casas}
            
        \vfill
            
        \vspace{0.8cm}
            
        \Large
        Despartamento de Ingeniería Electrónica y Telecomunicaciones\\
        Universidad de Antioquia\\
        Medellín\\
        Septiembre de 2020
            
    \end{center}
\end{titlepage}

%--\tableofcontents

\section{Taller Memoria}
%--Hola mundo \cite{einstein} \\

La memoria es el dispositivo electrónico que através de sus celdas, permite un almacenamiento temporal de la información (aplicaicones, archivos, datos), a fin de que esta sea más accesible al usuario. Almacena la información de manera temporal, mientras se ejecutan  las tareas, luego desaparace d ela memoria de forma permanente. \vspace{5mm} %5mm vertical space
\\
Hay varios tipos de memoria, de hecho todos los dispositivos electrónicos como los celulares, los televisores, las tablets o los computadores, tienen integrados varios tipos de memoria. En este documento se detallará las memorias de los computadores.  \vspace{5mm} %5mm vertical space
\\
Los computadores manejan varios tipos de memoria. Estos son:
\\
\begin{itemize}
  \item Memoria Cache L1, L2, L3
  \item Memoria RAM
  \item Memoria Virtual 
  \item Memoria Disco duro
\end{itemize}
\\
La memoria cache es la mas rapida del computador, sin embargo, esta compuesta de un sistema de circuitos complejo y núcleos, por o que es la más costosa, y generalmente tiene un espacio muy reducido. En ella hay tres capas de almacenamiento: L1 es el nivel más ráapido, donde se almacenan las aplicaicones más usadas, se encuentra en los núcleos. La siguiente es la L2, puede que no sea tan rápida como la L1 pero es muy rápida y también se almacena en los núcleos, pero posee mayor espacio. La siguiente es la L3, no se almacena en los núcleos, sino en alrededores a ellos pero tiene muchos mas espacio que L1 y L2. \vspace{5mm} %5mm vertical space

La memoria RAM es el espacio de computadora que genenralmente está vacía y sólo se llena con los archivos de datos que va a modificar los usuarios, y también con las aplicaciones que modificaráan dichos archivos.Una vez la información es modificada y se da la instrucción de cerrar, toda la información de la memoria desaparecerá para siempre, por eso es importante revisar si se deben guaradar los cambios, para que estos se incorporen al disco duro. \vspace{5mm} %5mm vertical space
\\
A través de impulsos eléctricos, el procesador manda la orden de desplegar ciertos recursos del disco duro, el cual tiene mayor capacidad de procesamiento que la memoria, pero menor capacidad de procesamiento. Estas ordenes las envia a través del bus de datos, ubicado en la placa madre. Una vez, los recursos estén en la memoria RAM, ocurre un intercambio de información continua entre la RAM y el procesador, el cual finalmente es quien ejecuta las tareas. La memoria RAM solo funciona como un escritorio temporal para almacenar los recursos que está utilizando el usuario, pero que originalmente fueron traídos del disco duro y que serán incorporandos nuevamiente al disco, una vez se finalice las tareas o se apague la máquina. \vspace{5mm} %5mm vertical space
La memoria virtual es una porción de disco duro, que es más velos que el resto del disco duro, pero inferior a la memoria RAM. En la memoria virtual se almacena aplicaciones que se usarán siempre al encender la máquina como el sistema operativo, esto permite establecer una mayor conexión entre el disco duro y la RAM, y que el rpoceso de cargue de las aplicaciones se demoren menos de segundos o segundos. \vspace{5mm} %5mm vertical space
El disco duro es el que posee mayor capacidad de almacenamiento, sin embargo, es lento en el procesamiento. En él, se almacenan todas las aplicaciones, datos, controladores, y archivos que contienen la máquina, los cuales son accedidos y llevados a la memoria RAM, a medida que el usuario lo solicite y dichas instrucciones son ejecutadas por la memoria RAM y el bus de datos que se ubica en la memoria madre. \vspace{5mm} %5mm vertical space
Lo que hace que una memoria sea más rapida que otra es su diseño de componentes electrónicos y microcircuitos, mientras más celdas de almacenamiento contengan las grillas de silicio de cada uno de lso componentes, mas velores serán, pues en cada celda se almacena un bit de información, asi más información viajaría por ciclo. En intercambio de información entre las memorias ocurre en ciclos por nanosegundos, medidos en Ghz, mientras mas Ghz cuente el computador o el procesador más ciclos de intercambio ocurren en un período de tiempo.






\end{document}
